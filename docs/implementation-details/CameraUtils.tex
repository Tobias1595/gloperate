\documentclass{article}

\usepackage{graphicx}
\usepackage{float}

\begin{document}

Given the parameters for a perspective projection we want to find a corresponding orthographic projection matrix and vice versa. The target projection should include the exact same portion of a plane perpendicular to the view direction. The plane is given by the distance to the camera.
The following explains the computation from perspective to orthographc. The other way is handled analogously.

A perspective projection is specified by a field of view (fovy), the aspect ratio and the distance to the near-/far plane (zNear, zFar). First we need to compute the distances of the edges to the center of the near plane for our perspective projection($l_p,r_p,b_p,t_p$). These get translated along the inverse z-axis up to the point of syncronisation and are used as input for the new orthographic projection($l_o,r_o,b_o,t_o$). The frustrum looks like this:
\begin{figure}[H]
  \centering
  \includegraphics[width=0.8\textwidth]{gl_projectionmatrix01}
  \caption{Perspective Frustum and Normalized Device Coordinates (NDC)}
\end{figure}

A point is projected on the near plane as shown here for the x coordinate:
\begin{figure}[H]
  \centering
  \includegraphics[width=0.4\textwidth]{gl_projectionmatrix03}
  \caption{Top View of Frustum; $x_e$ is eye space, $x_p$ is projected}
\end{figure}

This is done using the ratio of similar triangles:
\begin{equation}
    \label{similar_triangles}
    \frac{x_p}{x_e} = \frac{-n}{z_e}
\end{equation}




\end{document}
